% ----------------------------------------------------------------------------------------------------- %
% Arquivo LaTeX de exemplo de Projeto a ser apresentados à Ciência da Computação da UFT - Palmas
% 
% Versão 1.1:   Março 2016
%
% Criado por:   Tiago da Silva Almeida
% Revisado por: Tiago da Silva Almeida
%               Rafael Lima de Carvalho
%               Ary Henrique Morais de Oliveira
%
% http://uftex.sourceforge.net
% ----------------------------------------------------------------------------------------------------- %

\documentclass[tcc1,project]{uftex}	

\usepackage[num]{abntex2cite}	% Citações padrão ABNT

\begin{document}
  \title{Modelo de plano de trabalho para a disciplina de projeto de graduação II do curso de ciência da computação}
  \foreigntitle{Thesis Title}
  \author{Rafael}{Lima de Carvalho}
  \advisor{Prof.}{Rafae Lima Albuquerque}{de Carvalho}{D.Sc.}
  \advisor{Prof.}{Tiago da Silva}{Almeida}{D.Sc.}

  \department{CC}
  \date{04}{2016}

  \keyword{plano de trabalho}
  \keyword{\textit{templates}}
  \keyword{cronograma}
  
  \field{Leitura e prática de produção de textos}
  \field{Metodologia científica}

  \mainmatter
  
  \maketitle

  \begin{abstract}
O curso de ciência da computação vem passando por diversas transformações e atualizações desde sua fundação no ano de 2000. Com a última atualização do Projeto Pedagógico de Curso, a disciplina \textit{Projeto de graduação}, com carga horária de 240h, foi dividida, dando origem a duas novas disciplinas: \textit{Projeto de Graduação I} e \textit{Projeto de Graduação II}. 

O acadêmico matriculado na disciplina \textit{Projeto de Graduação I} possui o objetivo de construir as primeiras etapas do projeto de graduação, que geralmente inclui elementos como: decisão sobre o tema, nível de profundidade, levantamento bibliográfico e referencial teórico necessários para justificar a hipótese, além de resultados preliminares (opcional). Por outro lado, a missão do acadêmico matriculado no \textit{Projeto de Graduação II} possui como parte inicial da missão, o preenchimento do plano de trabalho que possui uma breve descrição e os objetivos de  seu trabalho, aliado a um cronograma atualizado das próximas etapas até a conclusão final de seu trabalho, que inclui a escrita e entrega da monografia.

Como bons brasileiros, a maioria dos acadêmicos matriculados nesta disciplina deixou para tirarem suas dúvidas a respeito do plano de trabalho, apenas na véspera da entrega. Diante deste fato, conjuntamente com as dúvidas enviadas por meio eletrônico, é proposto este modelo de plano de trabalho, de forma a auxiliar os acadêmicos do curso de ciência da computação na primeira etapa da disciplina de Projeto de Graduação II.
  \end{abstract}


% ----------------------------------------------------------------------------------------------------- %
% Capítulos do trabalho
% ----------------------------------------------------------------------------------------------------- %
\section*{Objetivos}
Os objetivos do presente projeto são: 
\begin{enumerate}
	\item Criar um modelo de plano de trabalho em Latex 
	\item Publicar o modelo no site da disciplina
	\item Disponibilizar em serviços online tais como \texttt{ShareLatex.com} e \texttt{Overleaf.com}
\end{enumerate}

\section*{Cronograma previsto de atividades \label{sec:crono}}

A descrição das atividades remanescentes está listada na Tabela \ref{tb:atividades}, enquanto que o cronograma é apresentado na Tabela \ref{tb:cronograma}.

%%%% INICIO ATIVIDADES PREVISTAS %%%%%%%%%%%%%%%%%

\setstretch{1} 
\begin{table}[!h]
  \centering
  \caption{Lista de atividades previstas.}\label{tb:atividades}
  \begin{tabular}{cp{9.4cm}}
    \hline \hline &\\[-0.4cm]
    {\bf Atividades} & \multicolumn{1}{c}{\bf Descrição} \\
    \hline
    &\\[-0.4cm]
    \textbf{A} &  Estudar o método CPTD. \\[0.2cm]
    \textbf{B} &  Transformar o problema de controle ótimo em sua variante do CPTD e assim obter um problema de otimização paramétrica.\\[0.2cm]
    \textbf{C} &  Modelar o Algoritmo Genético para otimizar o modelo obtido na etapa B.\\[0.2cm]
    \textbf{D} &  Modelar o Algoritmo de Nuvem de Partículas para otimizar o modelo obtido na etapa B. \\[0.2cm]
    \textbf{E} &  Executar os algoritmos em alguns cenários teste. \\[0.2cm]
    \textbf{F} &  Comparar os resultados.\\[0.2cm]
    \textbf{G} &  Embutir a restrição de conectividade.\\[0.2cm]
    \textbf{H} &  Repetir as etapas B a F com a variante proposta no item G. \\[0.2cm]
    \textbf{I} &  Comparar os resultados. \\[0.2cm]
    \textbf{J} &  Escrita da monografia. \\[0.2cm]
    \hline \hline
  \end{tabular}
\end{table}

%%% FIM ATIVIDADES PREVISTAS %%%%%%%%%%%%%%%%%


%%%%% INICIO DO CRONOGRAMA %%%%%%%%%%%%%%

\begin{table}[!h]
  \centering \fontsize{8}{12}%\tiny
  \caption{Cronograma de Atividades}\label{tb:cronograma}
  \begin{tabular}{|c|c|c|c|c|c|c|c|c|c|}
    \hline
    {\normalsize\bf Ano}  &\multicolumn{9}{c|}{\normalsize\bf 2014}\\
    \hline
 {\normalsize\bf Mês} &
 \multirow{2}*{\bf Jan}&\multirow{2}*{\bf Fev}&\multirow{2}*{\bf Mar}& \multirow{2}*{\bf Abr}&\multirow{2}*{\bf Mai}& \multirow{2}*{\bf Jun}& \multirow{2}*{\bf Jul}& \multirow{2}*{\bf Ago}& \multirow{2}*{\bf Set}\\
   \cline{1-1}
{\bf Atv.}    & & & & & & & & &  \\
\hline
{\normalsize\bf A} &$\surd$ & $\surd$ & $\surd$ & & & & & &  \\
\hline
{\normalsize\bf B} & &  & $\surd$ & $\surd$ & & & & & \\
\hline
%\hhline{>{\arrayrulecolor{black}}---->{\arrayrulecolor{black}}->{\arrayrulecolor{black}}------}
{\normalsize\bf C} & & & &$\surd$ & & & & &
\\
%\hhline{>{\arrayrulecolor{black}}----->{\arrayrulecolor{black}}-->{\arrayrulecolor{black}}----}
\hline
{\normalsize\bf D} &  &  &  &  & $\surd$& &  &  & \\
%\hhline{>{\arrayrulecolor{black}}------>{\arrayrulecolor{black}}->{\arrayrulecolor{black}}----}
\hline
{\normalsize\bf E} & & & &  &$\surd$ &$\surd$ & & & \\
%\hhline{>{\arrayrulecolor{black}}------->{\arrayrulecolor{black}}->{\arrayrulecolor{black}}---}
\hline
{\normalsize\bf F} & & & & & &$\surd$ & & & \\
% \hhline{>{\arrayrulecolor{black}}-------->{\arrayrulecolor{black}}-->{\arrayrulecolor{black}}-}
\hline
{\normalsize\bf G} & & & & & & $\surd$& $\surd$& & \\
% \hhline{>{\arrayrulecolor{black}}-------->{\arrayrulecolor{black}}-->{\arrayrulecolor{black}}-}
\hline
{\normalsize\bf H} & & & & & & & $\surd$&$\surd$ &\\
% \hhline{>{\arrayrulecolor{black}}-------->{\arrayrulecolor{black}}-->{\arrayrulecolor{black}}-}
\hline
{\normalsize\bf I} & & & & & & & & $\surd$& $\surd$ \\
\hline

{\normalsize\bf J} & & & & & &$\surd$ &$\surd$ &$\surd$ &$\surd$  \\
\hline
  \end{tabular}
\end{table}


% ----------------------------------------------------------------------------------------------------- %
% Bibliografia
% ----------------------------------------------------------------------------------------------------- %

\bibliography{projeto_exemplo}

\end{document}
