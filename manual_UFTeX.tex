% ----------------------------------------------------------------------------------------------------- %
% Manual da Classe UFTeX
% 
% Versão 1.1:   Março 2016
%
% Criado por:   Tiago da Silva Almeida
% Revisado por: Tiago da Silva Almeida
%               Rafael Lima de Carvalho
%               Ary Henrique Morais de Oliveira
%
% http://uftex.sourceforge.net
% ----------------------------------------------------------------------------------------------------- %

\documentclass[report]{uftex}
% ---- Esse comando cria o nome uftex estilizado
\newcommand\uftex{UF\TeX}
%\usepackage[numbers]{natbib}

\usepackage[num]{abntex2cite}                    % Citações padrão ABNT
% Define os textos da citação
%\renewcommand*{\backrefalt}[4]{}%
\renewcommand{\backrefpagesname}{Citado na(s) página(s):~}
% Texto padrão antes do número das páginas
\renewcommand{\backref}{}
% Define os textos da citação
\renewcommand*{\backrefalt}[4]{
	\ifcase #1 %
		Nenhuma citação no texto.%
	\or
		Citado na página #2.%
	\else
		Citado #1 vezes nas páginas #2.%
	\fi}%
    
% ----  Esse comandos são necessário no pré-ambulo para a impressão da lista de lista abreviatuas e de símbolos
\makelosymbols
\makeloabbreviations
% ---- Início do documento
\begin{document}
  % ---- Descrição do título do trabalho 
  \title{Manual para utilização e elaboração de trabalhos acadêmicos utilizando a classe \uftex .cls}
  % ---- Descrição do título do trabalho em idioma inglês, caso o trabalho seja escrito em ingles
  \foreigntitle{Thesis Title}
  % ---- Nome do autor ou autores do trabalho
  \author{Tiago}{da Silva Almeida}
  \author{Johnny}{Gomes Pereira}
  %\class{Sistemas Digitais}
  % ---- Nome do orientador do trabalho. O último campo representa o título do professor
  \advisor{Prof.}{Tiago}{da Silva Almeida}{M.Sc.}
  % ---- Descrição dos professores que compõem a banca examinadora
  \examiner{Prof.}{Nome do Primeiro Examinador Sobrenome}{D.Sc.}
  \examiner{Prof.}{Nome do Segundo Examinador Sobrenome}{Ph.D.}
  \examiner{Prof.}{Nome do Terceiro Examinador Sobrenome}{D.Sc.}
  % ---- Departamento representa o curso ao qual o trabalho está sendo apresentado. Descrito por meio de duas iniciais do curso
  \department{CC}
  % ---- Data da apresentação do trabalho
  \date{03}{2016}
  % ---- Palavras-chaves em português do trabalho
  \keyword{\LaTeX}
  \keyword{\uftex}
  \keyword{Trabalho de Conclusão de Curso}
  \keyword{Redação Científica}
  \keyword{Extensão Universitária}
  % ---- Palavras-chaves em inglês do trabalho
  \foreignkeyword{\LaTeX}
  \foreignkeyword{\uftex}
  \foreignkeyword{Bachelor Thesis}
  \foreignkeyword{Scientific Writing}
  \foreignkeyword{University Extension}
  % ---- Comando responsável por criar a capa do trabalho e/ou folha de resto conforme a configuração exigida
  \maketitle
  % ---- Esse comando marca o inicio dos elementos pré-textuais, e adiciona a numeração de páginas em algarismos romanos em caixa baixa
  \frontmatter
  % ---- Cria uma dedicatória ao trabalho. OPCIONAL
  %\dedication{A algu\'em cujo valor \'e digno desta dedicat\'oria.}
  % ---- Cria os agradecimentos do trabalho. OPCIONAL
  %\begin{acknowledgement}
  %Gostaria de agradecer a todos.
  %\end{acknowledgement}
  % ---- Cria o resumo em idioma escolhido pelo usuário, no caso em português. OBRIGATÓRIO
  \begin{abstract}
Este manual descreve a documentação da classe \uftex, arquivos distribuídos pelo projeto, exemplos de utilização e algumas dicas básicas para elaboração de textos acadêmicos. Esta classe é adequada para a escrita de trabalhos acadêmicos tais como: relatórios, pré-projetos, planos de trabalho e Trabalhos de Conclusão de Curso de acordo com as regras de formatação adotadas pelo Curso de Ciência da Computação da Universidade Federal do Tocantins, CUP. O conjunto minimalista de macros permite a seus usuários se concentrarem a maior parte de seus esforços na composição de texto em vez do \emph{layout} do documento.
  \end{abstract}
  % ---- Cria o resumo em idioma estrangeiro, no caso em inglês. OBRIGATÓRIO
  \begin{foreignabstract}
  In this work, we present ...
  \end{foreignabstract}
  \printlosymbols  
  \printloabbreviations
  % ---- Cria a lista de figuras. OPCIONAL
  %\listoffigures
  % ---- Cria a lista de tabelas. OPCIONAL
  %\listoftables 
  % ---- Cria o sumário. OBRIGATÓRIO
  \tableofcontents % sumário
% --- Marca o inicio dos elementos textuais. Capítulos.
\mainmatter
% ---- Defino o espaçamento de um e meio centímetros
\onehalfspacing
% ----------------------------------------------------------------------------------------------------- %
% Capítulos do trabalho
% ----------------------------------------------------------------------------------------------------- %
\chapter{Introdução}
\label{sec:introducao}
% ALTERAR texto desta seção
\noindent Escrever documentos acadêmicos pode ser uma tarefa trabalhosa quando os autores têm que preparar os seus manuscritos rigorosamente respeitando as regras de formatação imposta pela instituição de ensino ou mesmo por agentes externos, como por exemplo, a ABNT \abbrev{ABNT}{Associação Brasileira de Normas Técnicas}. Visando, automatizar esse processo, deixando que o estudante foque mais no desenvolvimento do trabalho, ao invés da despender muito tempo com a formatação do mesmo. Atualmente, os TCCs apresentados ao Curso de Ciência da Computação da Universidade Federal do Tocantins (CComp/UFT\abbrev{UFT}{Universidade Federal do Tocantins}) é editado em \LaTeX\, e essa demanda motivou o criação do projeto \uftex, que tenta facilitar e incentivar o uso do \LaTeX\ no âmbito CComp/UFT. \cite{JW82}

A classe \uftex\ foi projetada para ser clara e sucinta. Ela permite a criação de relatórios, pré-projetos, planos de trabalho e Trabalhos de Conclusão de Curso de uma forma simples e automática. O objetivo principal da classe \uftex\ é manter autores estritamente focados na composição de texto sem se preocupar com margens, espaçamento entre linhas, tamanho do papel, o alinhamento vertical e horizontal, etc.

%% ------------------------------------------------------------------------- %%
\section{Licença}
\label{sec:Licenca}
% SUPRIMIR esta seção?
\noindent Cada arquivo pertencente a este pacote contém um aviso de \emph{copyright}. A sua utilização está protegida pela GNU \emph{General Public License} (GPL) versão 3, de modo que os usuários são livres para copiar, distribuir ou modificar o código-fonte, entre outros atos abrangidos por esta licença. Para ver o texto completo da licença GNU GPL, vá para o arquivo COPYING anexado a este pacote. 

%% ------------------------------------------------------------------------- %%
\section{Suporte}
\label{sec:suporte}
% SUPRIMIR esta seção?
\noindent Nós mantemos uma lista de discussão onde os usuários podem enviar perguntas, comentários e bugs. Mais detalhes podem ser encontrados em \url{http://uftex.sourceforge.net}


%% ------------------------------------------------------------------------- %%
\chapter{Descrição dos comandos}
\section{Opções da Classe}
\label{sec:opcoes-da-classe}

\noindent Esta classe contém uma série de configurações que visam facilitar o trabalho do aluno no momento da produção no \LaTeX. 

A classe \uftex\ vem pré-configurada para cinco tipos de documentos distintos, são eles:

\begin{itemize}
	\item
	% -------- Dedicar uma seção só para falar deste ambiente
	\verb+report+ -- usada de forma genérica. Pode ser utilizada em relatórios de experimentos, aulas, trabalhos, etc. Esse comando cria somente permite que seja criada somente uma capa simples e a inclusão de vários autores ao mesmo documento.
	% ---------- Dedicar seção para cada um destes e referenciá-la com os respectivos itens
	\item
	% -------- Dedicar uma seção só para falar deste ambiente
	\verb+tcc+ -- devido à mudança de grade do curso e na uma tentativa de generalizar à outros curso o projeto, esse opção cria uma capa formatada e folha de rosto ao documento, com o rótulo de Trabalho de Conclusão de Curso.
	item
	\item 
	\verb+tcc1+ -- devido à mudança de grade do curso e na uma tentativa de generalizar à outros curso o projeto, esse opção cria uma capa formatada e folha de rosto ao documento, com o rótulo de Trabalho de Conclusão de Curso I.
	\item 
	\verb+tcc2+ -- devido à mudança de grade do curso e na uma tentativa de generalizar à outros curso o projeto, esse opção cria uma capa formatada e folha de rosto ao documento, com o rótulo de Trabalho de Conclusão de Curso II.
	\item 
	\verb+project+ -- utilizada para estilização do documento para criação de um pré-projeto do trabalho. Essa opção deve ser utilizada em conjunto com a opção \verb+tcc+, \verb+tcc1+ ou \verb+tcc2+.
\end{itemize}

Você pode especificar a opção de documento escolhida através do \begin{verbatim}
\documentclass[opções]{uftex}
\end{verbatim}

É importante destacar que as opções: \verb+report+, \verb+tcc+, \verb+tcc1+ e \verb+tcc2+ são conflitantes e não devem ser utilizadas em conjunto.

%% ------------------------------------------------------------------------- %%
%% ------------ Nesta seção é descrito a correta utilização dos pacotes necessários para uso da classe
%% ------------ uftex.cls 
%% ------------ Trate-se de uma seção facilmente descartável.
%% ------------------------------------------------------------------------- %% 
\section{Pacotes}
\label{sec:pacotes}

\noindent Ao utilizar a classe \uftex\ é necessário ater-se a alguns detalhes referentes aos pacotes. Pois, na implementação do mesmo é necessário uma lista de pacotes para o correto funcionamento da classe. Para usuários de sistemas GNU/Linux será necessário a instalação manual de cada um destes pacotes via terminal. Usuários Windows podem fazer a instalação desses pacotes através do próprio ambiente de desenvolvimento que estiverem utilizando. A lista de pacotes necessários é descrita no Apêndice \ref{ape:importacao-de-pacotes}.\abbrev{ABNT}{Brasileira de Normas Técnicas}

A classe foi projetada para que poucos ou nenhum pacote precisasse ser incorporado ao preâmbulo dos trabalhos. Entretanto, caso houver necessidade, sinta-se á vontade para adicionar funcionalidades ao trabalho através da importação de outros pacotes com o comando:

\begin{verbatim}
\usepackage[opções do pacote]{nome-do-pacote}
\end{verbatim}


%% ------------------------------------------------------------------------- %%
\section{Estrutura Visual}
\label{sec:contribucoes}

\noindent Os documentos produzidos a partir da classe \uftex\ devem conter três partes: pré-textuais, textuais e pós-textuais. Cada uma dessas partes são iniciadas chamando seu macro correspondente \verb+\frontmatter+, \verb+ \mainmatter+ ou \verb+\backmatter+. Os \emph{pré-textuais} de um documento consistem em capa, folha de rosto, dedicatória, agradecimentos, resumos seguidos de palavras chaves, resumo em língua estrangeira, lista de abreviaturas, lista de símbolos, lista de algoritmos, listas de figuras, listas de tabelas e sumário. A parte principal ou textuais é composta apenas por capítulos, com suas seções e subseções, enquanto a parte pós-textual consiste de referências bibliográficas, apêndices e anexos.

 Você deve chamar o macro \verb+\frontmatter+ imediatamente após o \verb+\maketitle+. O comando \verb+\mainmatter+ vem logo antes do primeiro capítulo, e \verb+\backmatter+ deve ser digitado antes das referências bibliográficas.
%% ------------------------------------------------------------------------- %%
\section{Capa}
\label{sec:capa}

\noindent Este elemento é automaticamente construído pelo comando \verb+\maketitle+. A opção de documento \verb+project+ inibe a construção do elemento capa. \cite{JW82}

Obrigatoriamente, para a construção do elemento capa, devem ser inseridos os seguintes comandos:

\begin{itemize}
\item \verb+\author{}{}+ -- O comando \verb+\author+ foi redefinido. Aqui, ele leva dois argumentos: o primeiro nome do autor e o sobrenome, por exemplo, \verb+\author{Primeiro nome}{Sobrenome}+. Se a opção escolhida for \verb+report+, mais de um autor poderá ser adicionado ao documento.
\item \verb+\title{}+ -- O comando \verb+\title+  é usados para inserir os títulos de sua monografia em língua materna.
\item \verb+\foreigntitle{}+ -- O comando \verb+\foreigntitle+ é utilizado para colocar o título da monografia em língua estrangeira. Utilizado somente para o caso do aluno desejar escrever seu trabalho em outro idioma, por exemplo, em inglês. Caso contrário não é necessário utilizá-lo.
\item \verb+\advisor{}{}{}{}+ -- Comando utilizado para acrescentar o nome do orientador do trabalho. Ele é dividido em quatro campos: profissão, primeiro nome, sobrenome e titulação, conforme:

\begin{verbatim}
	\advisor{Prof.}{Nome do Primeiro Orientador}{Sobrenome}{Ph.D.}
	\advisor{Prof.}{Nome do Segundo Orientador}{Sobrenome}{D.Sc.}
\end{verbatim}

Mais de um orientador pode ser adicionado, para o caso de trabalhos co-orientados. Se for escolhida a opção \verb+report+ não é necessário a utilização desse comando.
\item \verb+\department{}+ -- Quanto ao departamento, a princípio estão cadastradas as seguintes abreviaturas: EC (Engenharia Civil), EE (Engenharia Elétrica), EA (Engenharia Ambiental), CC (Ciência da Computação) e AL (Engenharia de Alimentos). Você deve especificar o seu departamento usando uma das abreviaturas acima, por exemplo, \verb+\department{CC}+.
\item \verb+\date{}{}+ -- Este comando é usado para definir o mês e ano da defesa. Por exemplo, Janeiro de 2016 deve ser inserido como \verb+\date{01}{2016}+.
\item \verb+\field{}+ -- Esse comando adiciona os campos da área de pesquisa do trabalho a ser desenvolvido de acordo com a classificação de área da ACM. Esse comando é utilizado somente quando a opção de classe \verb+project+ for escolhida para elaboração do documento.
\item \verb+\class{}+ -- Esse comando adiciona o nome de uma disciplina ao cabeçalho da capa. Esse comando é utilizado somente quando a opção de classe \verb+report+ for escolhida para elaboração do documento e é um campo opcional. A sua não utilização implica somente em não aparecer o nome da disciplina no cabeçalho.
\end{itemize}

%% ------------------------------------------------------------------------- %%
\section{Folha de rosto}
\label{cap:folhaderosto}

\noindent Este elemento também é construída a partir do \verb+\maketitle+, se as opções escolhidas no documento forem: \verb+tcc+, \verb+tcc1+ ou \verb+tcc2+. 
	
Os orientadores não são necessariamente membros da banca examinadora do TCC. Assim, é necessário digitar os nomes de todos os avaliadores usando o comando \verb+\examiner{}{}{}+. Os nomes dos examinadores são inseridos da seguinte forma:

\begin{verbatim}
	\examiner{Prof.}{Nome do Primeiro Examinador Sobrenome}{Ph.D.}
	\examiner{Prof.}{Nome do Segundo Examinador Sobrenome}{D.Sc.}
	\examiner{Prof.}{Nome do Terceiro Examinador Sobrenome}{D.Sc.}
\end{verbatim}	

\section{Antecedendo \emph{Resumo} e \emph{Abstract}}

As palavras-chave devem descrever as áreas de concentração de seu trabalho. Essas informações serão utilizadas na criação do resumo. Você deve fornecê-las como se segue:

\begin{verbatim}
	\keyword{Primeira palavra-chave}
	\keyword{Segunda palavra-chave}
	\keyword{Terceira palavra-chave}
\end{verbatim}


As palavras chaves em língua estrangeira também devem ser descritas para criação do \emph{Abstract}, utilizando os comandos:

\begin{verbatim}
	\foreignkeyword{First keyword}
	\foreignkeyword{Second keyword}
	\foreignkeyword{Third keyword}
\end{verbatim}

{\color{red}Lembre-se que todos os nomes devem ser dados antes do comando \verb+\maketitle+.}

%% ------------------------------------------------------------------------- %%
\section{Dedicatória (Opcional)}
\label{sec:dedicatoria}

Este comando foi adicionado por conveniência. O texto de entrada é colocado no lado inferior direito de uma página em branco. Deve ser enfatizado e em tamanho normal. A forma correta de utilizar esta macro é:
\begin{verbatim}\dedication{A alguém cujo valor é digno desta dedicatória.}\end{verbatim} e a mesma deverá vir logo abaixo do comando \verb+\frontmatter+.


%% ------------------------------------------------------------------------- %%
\section{Resumo e \emph{Abstract}}
\label{sec:resumos}

O resumo e \emph{abstract} devem estar em uma página cada, com em torno de 250 palavras. É recomendável que eles tenham apenas um parágrafo longo. Eles devem ser definidos dentro dos ambientes \verb+\abstracts+ e \verb+\foreignabstract+. Por exemplo:

\begin{verbatim}
\begin{abstract}
Algum texto...    
\end{abstract}
\end{verbatim}

E

\begin{verbatim}
\begin{foreignabstract}
Algum texto...    
\end{foreignabstract}
\end{verbatim}


%% ------------------------------------------------------------------------- %%
\section{Lista de Símbolos e Abreviaturas (Opcional)}
\label{sec:simbolos-e-abreviaturas}

\noindent As listas de símbolos e abreviaturas são opcionais, embora altamente recomendadas. \cite{JW82}
É uma boa prática definir um símbolo/abreviatura em sua primeira ocorrência no texto. Para definir um símbolo de uso \verb+\symbl{Símbolo}{Definição do Símbolo}+, e para abreviaturas \verb+\abbrev{Abreviatura}{Abreviatura Definição}+.
É interessante destacar que estes comandos não provocam alteração no lugar onde são escritos, ou seja, só aparecem na lista de símbolos e abreviaturas. % explicar melhor isso aqui.

Estas listas são lexicograficamente classificadas usando o programa \emph{MakeIndex}, que é parte de qualquer implementação \LaTeX. \emph{MakeIndex} precisa de dois comandos para criar uma lista final ordenada: um que gera uma lista de entradas e outro que indica a posição onde a lista será impressa. Para gerar as listas de símbolos e abreviaturas, a classe \uftex\ fornece os comandos \verb+\makeloabreviations+ e \verb+\makelosymbols+, respectivamente. Eles devem ser chamados no preâmbulo do documento. Os comandos \verb+\printlosymbols+ e \verb+\printloabbreviations+ tem que ser invocados no ponto onde você quer que estas listas apareçam, por exemplo, seguindo a lista de tabelas como por exemplo:

\singlespacing
\footnotesize
\begin{verbatim}
\documentclass[tcc]{uftex}
% ----  Esse comandos são necessário no pré-ambulo para a impressão da
%lista de lista abreviatuas e de símbolos
\makelosymbols
\makeloabbreviations
% ---- Início do documento
\begin{document}
  .
  .
  .
  % ---- Comando responsável por criar a capa do trabalho e/ou folha de
  %resto conforme a configuração exigida
  \maketitle
  % ---- Esse comando marca o inicio dos elementos pré-textuais, e
  %adiciona a numeração de páginas em algarismos romanos em caixa baixa
  \frontmatter
  % ---- Cria uma dedicatória ao trabalho. OPCIONAL
  \dedication{A algu\'em cujo valor \'e digno desta dedicat\'oria.}
  % ---- Cria os agradecimentos do trabalho. OPCIONAL
  \begin{acknowledgement}
  Gostaria de agradecer a todos.
  \end{acknowledgement}
  % ---- Cria o resumo em idioma escolhido pelo usuário, no caso em
  %português. OBRIGATÓRIO
  \begin{abstract}
  Algum texto ...
  \end{abstract}
  % ---- Cria o resumo em idioma estrangeiro, no caso em inglês.
  %OBRIGATÓRIO
  \begin{foreignabstract}
  In this work, we present ...
  \end{foreignabstract}
  \printlosymbols  
  \printloabbreviations
  % ---- Cria a lista de figuras. OPCIONAL
  \listoffigures
  % ---- Cria a lista de tabelas. OPCIONAL
  \listoftables 
  % ---- Cria o sumário. OBRIGATÓRIO
  \tableofcontents % sumário
% --- Marca o inicio dos elementos textuais. Capítulos.
\mainmatter
% ---- Defino o espaçamento de um e meio centímetros
\onehalfspacing
% --------------------------------------------------------------------- %
% Capítulos do trabalho
% --------------------------------------------------------------------- %
\chapter{Introdução}
.
.
.
\backmatter 
\singlespacing   % espaçamento simples
% --------------------------------------------------------------------- %
% Bibliografia
% --------------------------------------------------------------------- %
\bibliographystyle{plainnat} % citação bibliográfica alpha
\bibliography{exemplo}

% --------------------------------------------------------------------- %
% Anexos
% --------------------------------------------------------------------- %
\appendix

\end{document}
\end{verbatim}
\onehalfspacing
\normalsize

Uma vez que você compila o \texttt{latex}, ele criará dois arquivos com extensões \texttt{abx} e \texttt{syx}, que contêm dados de entrada \emph{MakeIndex}. Eles devem ser processados com \texttt{makeindex} a fim de obter as listas produzidas corretamente, redirecionando a saída para arquivos com extensão \texttt{lab} e \texttt{los} respectivamente:

\begin{verbatim}
	makeindex -s uftex.ist -o exemplo.lab exemplo.abx
	makeindex -s uftex.ist -o exemplo.los exemplo.syx
\end{verbatim}

Observe a opção \texttt{-s} para especificar o estilo \emph{uftex.ist}. Agora, compile o \texttt{latex} duas vezes para obter as referências e está feito. % explicar melhor isso aqui. Especificar com mais clareza a ordem em que os comandos devem ser feitos, ou seja, explicar melhor o processo.

%% ------------------------------------------------------------------------- %%
\chapter{Elaboração do documento}
\section{Dicas Úteis}
\label{sec:dicas}

\begin{itemize}
\item \textbf{Parágrafos}. Todos os parágrafos no inicio de cada Capítulo ou Seção devem ser iniciados sem indentação utilizando o comando \verb+\noindent+. \cite{JW82}

\item \textbf{Citações}. Para citações longas com mais de três linhas é possível utilizar o aperfeiçoamento do ambiente \verb+\quote+, como por exemplo:

\begin{verbatim}
    \begin{quote}
    ``Minha citação''
    \end{quote}
\end{verbatim}

Porém, esse recurso deve ser utilizado com muito cuidado para evitar situação de plágio. 

\item \textbf{Imagens}. O formato de imagem padrão do \LaTeX é a \emph{Encapsulated PostScript} (EPS). Se você usar PDF \LaTeX, o formato padrão se torna o PDF, mas você pode igualmente carregar arquivos PNG. Para tal, você deve digitar o nome do arquivo de imagem sem extensão, por exemplo, 

\begin{verbatim}
\begin{figure}
  \includegraphics[dimensões]{nome-do-arquivo}
  \caption{Legenda.}\label{chave_para_refencia_cruzada}
\end{figure}
\end{verbatim}

e o pdflatex irá procurar em primeiro lugar um arquivo chamado \textit{nome-do-arquivo.pdf} e depois para \textit{nome-do-arquivo.png}.

% ---------- \paragraph{}
\item \textbf{Fontes}. A fonte padrão em \LaTeX\ é o \emph{Computer Modern}. Se você quiser uma versão melhorada, considere usar o pacote \emph{lmodern}. Para usar o \emph{Times}, é recomendado carregar o pacote \emph{mathptmx}. Há também uma versão melhorada da \emph{Times} disponível com o pacote \emph{tgtermes}. Você ainda pode usar o tipo de letra \emph{Arial} com o pacote \emph{uarial}. % Mas, acho que é melhor não modificar estas configurações, certo?

%% --------- \paragraph{}
\item \textbf{Hyperref}. Ao trabalhar com PDFs, há a possibilidade de adicionar informações extras para o arquivo como o nome do autor, título do documento, assunto, palavras-chave, etc. Isso é feito com facilidade através do pacote \emph{hyperref}. Também é útil para permitir \emph{hiperlinks}. Felizmente, a classe \uftex\ vai fazer isso automaticamente se o pacote \emph{hyperref} for carregado. % Para maiores informações consulte \ref{sec:pacotes}.

%% --------- \paragraph{}
\item  \textbf{Impressão}. Para que seu trabalho seja impresso corretamente, você deve garantir que qualquer opção de escala de página (por exemplo, a adequação ou encolhimento para área de impressão) não esteja habilitado. Este tipo de opção, muitas vezes vem em diálogo de impressão de softwares de visualização de documentos. % 
\end{itemize}

%% ------------------------------------------------------------------------- %%
\section{Referências Bibliográficas}
\label{sec:referencias}

\noindent Sabe-se que os dados bibliográficos podem ser facilmente mantidos com o auxílio do BibTeX. A forma correta de utilizar este recurso é  incluindo suas referências BibTeX sem a extensão bib, como no exemplo a seguir:

\begin{verbatim}
	\bibliographystyle{plainnat} % citação bibliográfica alpha
	\bibliography{exemplo}
\end{verbatim}


%% ------------------------------------------------------------------------- %%
\section{Algumas Referências}
\label{sec:algumas_referencias}

\noindent É muito recomendável a utilização de arquivos \emph{bibtex} para o gerenciamento de referências a trabalhos. Exemplos de referências com a tag:

\begin{itemize}
\item @book: \cite{JW82}.
{\scriptsize\begin{verbatim}
@book{JW82,
 author   = {Richard A. Johnson and Dean W. Wichern},
 title    = {Applied Multivariate Statistical Analysis},
 publisher= {Prentice-Hall},
 year     = {1983}
}
\end{verbatim}}

\item @article: \cite{MenaChalco08}.
{\scriptsize\begin{verbatim}
@article{MenaChalco08,
 author   = {Jesús P. Mena-Chalco and Helaine Carrer and Yossi Zana and 
            Roberto M. Cesar-Jr.},
 title    = {Identification of protein coding regions using the modified 
            {G}abor-wavelet transform},
 journal  = {IEEE/ACM Transactions on Computational Biology and Bioinformatics},
 volume   = {5},
 pages    = {198-207},
 year     = {2008},
}
\end{verbatim}}

\item @inProceedings: \cite{alves03:simi}.
{\scriptsize\begin{verbatim}
@inproceedings{alves03:simi,
 author   = {Carlos E. R. Alves and Edson N. Cáceres and Frank Dehne and 
            Siang W. Song},
 title    = {A Parallel Wavefront Algorithm for Efficient Biological 
            Sequence Comparison},
 booktitle= {ICCSA '03: The 2003 International Conference on Computational 
            Science and its Applications},
 year     = {2003},
 pages    = {249-258},
 month    = May,
 publisher= {Springer-Verlag}
}
\end{verbatim}}

\item @incollection: \cite{bobaoglu93:concepts}.
{\scriptsize\begin{verbatim}
@InCollection{bobaoglu93:concepts,
 author   = {Ozalp Babaoglu and Keith Marzullo},
 title    = {Consistent Global States of Distributed Systems: Fundamental 
             Concepts and Mechanisms},
 editor   = {Sape Mullender},
 booktitle= {Distributed Systems},
 edition  = {segunda},
 year     = {1993},
 pages    = {55-96}
}
\end{verbatim}}

\item @conference: \cite{bronevetsky02}.
{\scriptsize\begin{verbatim}
@Conference{bronevetsky02,
 author   = {Greg Bronevetsky and Daniel Marques and Keshav Pingali and 
            Paul Stodghill},
 title    = {Automated application-level checkpointing of {MPI} programs},
 booktitle= {PPoPP '03: Proceedings of the 9th ACM SIGPLAN Symposium on Principles
            and Practice of Parallel Programming},
 year     = {2003},
 pages    = {84-89}
}
\end{verbatim}}

\item @phdThesis: \cite{garcia01:PhD}.
{\scriptsize\begin{verbatim}
@PhdThesis{garcia01:PhD,
 author   = {Islene C. Garcia},
 title    = {Visões Progressivas de Computações Distribuídas},
 school   = {Instituto de Computação, Universidade de Campinas, Brasil},
 year     = {2001},
 month    = {Dezembro}
}
\end{verbatim}}

\item @mastersThesis: \cite{schmidt03:MSc}.
{\scriptsize\begin{verbatim}
@MastersThesis{schmidt03:MSc,
 author   = {Rodrigo M. Schmidt},
 title    = {Coleta de Lixo para Protocolos de \emph{Checkpointing}},
 school   = {Instituto de Computação, Universidade de Campinas, Brasil},
 year     = {2003},
 month    = Oct
}
\end{verbatim}}

\item @techreport: \cite{alvisi99:analysisCIC}.
{\scriptsize\begin{verbatim}
@Techreport{alvisi99:analysisCIC,
 author   = {Lorenzo Alvisi and Elmootazbellah Elnozahy and Sriram S. Rao and
            Syed A. Husain and Asanka Del Mel},
 title    = {An Analysis of Comunication-Induced Checkpointing},
 institution= {Department of Computer Science, University of Texas at Austin},
 year     = {1999},
 number   = {TR-99-01},
 address  = {Austin, {USA}}
}
\end{verbatim}}

\item @manual: \cite{CORBA:spec}.
{\scriptsize\begin{verbatim}
@Manual{CORBA:spec,
 title    = {{CORBA v3.0 Specification}},
 author   = {{Object Management Group}},
 month    = Jul,
 year     = {2002},
 note     = {{OMG Document 02-06-33}}
}
\end{verbatim}}

\item @Misc: \cite{gridftp}.
{\scriptsize\begin{verbatim}
@Misc{gridftp,
 author   = {William Allcock},
 title    = {{GridFTP} protocol specification. {Global Grid Forum}
            Recommendation ({GFD}.20)},
 year     = {2003}
}
\end{verbatim}}

\item @misc: para referência a artigo online \cite{fowler04:designDead}.
{\scriptsize\begin{verbatim}
@Misc{fowler04:designDead,
 author   = {Martin Fowler},
 title    = {Is Design Dead?},
 year     = {2004},
 month    = May,
 note     = {Último acesso em 30/1/2010},
 howpublished= {\url{http://martinfowler.com/articles/designDead.html}},
}
\end{verbatim}}

\item @misc: para referência a página web \cite{FSF:GNU-GPL}.
{\scriptsize\begin{verbatim}
@Misc{FSF:GNU-GPL,
 author   = {Free Software Foundation},
 title    = {GNU general public license},
 note     = {Último acesso em 30/1/2010},
 howpublished= {\url{http://www.gnu.org/copyleft/gpl.html}},
}
\end{verbatim}}

\end{itemize}

\backmatter 
\singlespacing   % espaçamento simples
% ----------------------------------------------------------------------------------------------------- %
% Bibliografia
%\bibliographystyle{plainnat} % citação bibliográfica alpha
\bibliography{manual_uftex}

\appendix

\chapter{Importação de Pacotes}
\label{ape:importacao-de-pacotes}

\noindent A seguir, vê-se alguns dos principais pacotes a serem instalados:

\singlespacing
	\begin{verbatim}
	
	% --- Idioma do texto, a acentuação pode ser escrita normalmente
	fontenc.sty
	
	% --- Para configurar a linguagem do documento, hifenização, nomes, etc...
	babel.sty
	
	% --- Para configurar a codificação do arquivo de entrada
	inputenc.sty
	
	% --- Utilização da marca d'agua
	pdfpages.sty
	
	% --- Usamos arquivos pdf/png como figuras
	wallpaper.sty
	
	% --- Pacote padrão para incluir figuras, usando comandos;
	graphicx.sty
	
	% --- Espaçamento flexível
	setspace.sty
	
	% --- Indentação do primeiro parágrafo
	indentfirst.sty
	
	% --- Acrescentamos a bibliografia/indice/conteudo no Table of Contents
	tocbibind.sty
	
	% --- Múltiplas colunas e linhas em tabelas
	multirow.sty		
	
	% --- Utilização de cores em tabelas
	colortbl.sty
	
	% --- Rotação e objeto flutuante, como figuras e tabelas
	rotating.sty
	
	% --- Referenciar o número de páginas no documento
	lastpage.sty
	
	% --- Para incluir figuras lado a lado no latex
	subfigure.sty

	% --- Permite texto ao redor de tabelas
	wrapfig.sty
	
	% --- Para utilização e formatação de código fonte em diversas linguagens
	listings.sty
	
	% --- cabeçalhos dos títulos: menores e compactos
	titlesec.sty 
	
	% --- Formatação das captions de figuras e tabelas
	caption.sty
	
	% --- Margens
	geometry.sty 
	
	% --- Links em preto
	hyperref.sty
	
	% --- soluciona o problema com o hyperref e capitulos
	hypcap.sty
	
	% --- Estilo de referências bibliográficas
	natbib.sty
	
	% --- Utilização de cores
	color.sty	
	
	\end{verbatim}

\end{document}
